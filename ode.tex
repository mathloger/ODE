\documentclass{ctexart}
\ctexset{
	section = {
		titleformat = \raggedright,
		name = {},
		number = \chinese{section}
	}
}
\usepackage[a4paper,top=2.54cm,bottom=2.54cm,left=3.18cm,right=3.18cm]{geometry}
\usepackage{amsmath,amsfonts,amssymb,amsthm}
\usepackage{tikz}
\usepackage{multirow}
\usepackage{array}
\usepackage{fancyhdr}
\usepackage{lastpage}

\pagestyle{fancy}
\fancyhf{}
\cfoot{【第 \thepage 页~~共 \pageref{LastPage} 页】}
\renewcommand{\headrulewidth}{0pt}
\renewcommand{\labelenumii}{\Alph{enumii}.}

\begin{document}
	\begin{center}
		\textbf{\LARGE 华南农业大学}\par
		\vspace{8pt}
		\textbf{\LARGE 2021--2022学年第一学期期末考试试卷}
	\end{center}
	
	\begin{center}
		\begin{tabular}{m{0.33\textwidth} m{0.25\textwidth} m{0.28\textwidth}}
			授课课时:48学时 &考试时长:120分钟\\
			课程名称:\textbf{常微分方程} 
			& \multicolumn{2}{l}{适用对象:2019级信息与计算科学1-2班}\\
			试卷命题人:\underline{加绒}
			&试卷审核人:\underline{加绒}
		\end{tabular}
	\end{center}
	\hrule height 2pt
	\vspace{0.5cm}
	\textbf{注:带$\divideontimes$号的为我记不太清楚的题目!对此一般作化简处理,或替换为同类型的题目。}
	\section{选择题(共15分,每题3分)}
	\begin{enumerate}
		\item 下列微分方程为二阶线性方程的是( ).
			\begin{enumerate}
				\item $\dfrac{\mathrm{d}^2 y}{\mathrm{d}x^2}-(\dfrac{\mathrm{d}y}{\mathrm{d}x})^2+12xy=0$
				\item 2 $(\dfrac{\mathrm{d}y}{\mathrm{d}x})^2+x\dfrac{\mathrm{d}y}{\mathrm{d}x}-3y^2=0$
				\item 3 $\dfrac{\mathrm{d}^2 y}{\mathrm{d}x^2}+x\dfrac{\mathrm{d}y}{\mathrm{d}x}+x=\sin y$
				\item $x\dfrac{\mathrm{d}^2y}{\mathrm{d}x^2}-5\dfrac{\mathrm{d}y}{\mathrm{d}x}+3xy=\sin x$
			\end{enumerate}
		
		\item 微分方程$y''+p(x)y'+q(x)y=f(x)$有三个特解$y_1=e^x, y_2=xe^x, y_3=x^2e^x$,则微分方程的通解为( ).
			\begin{enumerate}
				\item $y=(C_1+C_2x)xe^x+e^x$
				\item $y=(C_1+C_2x)xe^x-e^x$
				\item $y=(C_1+C_2x+C_3x^2)e^x+$
				\item $y=(C_1+C_2x)xe^x+(1-C_1-C_2)e^x$
			\end{enumerate}
		
		\item $\divideontimes$求方程$\dfrac{\mathrm{d}^4 x}{\mathrm{d}t^4}-x=0$的基本解组( ).
			\begin{enumerate}
				\item $1,-1,i,-i$
				\item $1,0,i,-i$
				\item $e^t,e^{-t},\cos t,\sin t$
				\item $e^t,1,\cos t,\sin t$
			\end{enumerate}
		
		\item 李普希茨条件是一阶线性微分方程的解存在唯一的( )条件.
			\begin{enumerate}
				\item 充分
				\item 必要
				\item 充分必要
				\item 必要不充分
			\end{enumerate}
		
		\item 微分方程$N(x,y)\mathrm{d}x+M(x,y)\mathrm{d}y=0$存在积分因子$\mu$的充分必要条件为( ).
			\begin{enumerate}
				\item $\dfrac{\partial (\mu M)}{\partial x}=\dfrac{\partial (\mu N)}{\partial y}$
				\item $\dfrac{\partial (\mu M)}{\partial y}=\dfrac{\partial (\mu N)}{\partial x}$
				\item $\dfrac{\partial (\mu M)}{\partial x}\neq\dfrac{\partial (\mu N)}{\partial y}$
				\item $\dfrac{\partial (\mu M)}{\partial y}\neq\dfrac{\partial (\mu N)}{\partial x}$
			\end{enumerate}
		\vspace{1cm}
		\textbf{(以下拓展题目留给读者复习用)}
		\item 以下各函数组在它们相应的定义区间内线性相关的有( ).
			\begin{enumerate}
				\item $\sin 2t,\cos t,\sin t$
				\item $\cos 2t,1,\cos^2 t$
				\item $1,x,x^2$
				\item $e^t,te^t,t^2e^t$
			\end{enumerate}
		
		\item 微分方程$M(x,y)\mathrm{d}x+N(x,y)\mathrm{d}y=0$存在只与$x$有关的积分因子的充分必要条件为( ).
		\begin{enumerate}
			\item $\dfrac{\dfrac{\partial M}{\partial y}-\dfrac{\partial N}{\partial x}}{N}=\psi(y)$
			\item $\dfrac{\dfrac{\partial M}{\partial y}-\dfrac{\partial N}{\partial x}}{-N}=\psi(y)$
			\item $\dfrac{\dfrac{\partial M}{\partial y}-\dfrac{\partial N}{\partial x}}{M}=\psi(x)$
			\item $\dfrac{\dfrac{\partial M}{\partial y}-\dfrac{\partial N}{\partial x}}{N}=\psi(x)$
		\end{enumerate}
	\end{enumerate}
	
	\section{填空题(共15分,每题3分)}
	\begin{enumerate}
		\item 方程$xy'+y=xy^2\ln x$通过变换\underline{\makebox[6em]{ }}可化为线性方程.
		
		\item $\divideontimes$ 方程$\dfrac{\mathrm{d}y}{\mathrm{d}x}=\dfrac{x-y+1}{x+y-3}$通过变换\underline{\makebox[6em]{ }}可化为齐次方程.
		
		\item $\divideontimes$已知某一四阶实常系数线性齐次方程只有特征根$0,\pm4i$,则还原该微分方程为\underline{\makebox[6em]{ }}.
		
		\item $\divideontimes$(完全记不清了)方程$y''+4y'+4y=0$的基本解组为\underline{\makebox[6em]{ }}.
		
		\item 二阶欧拉方程$x^2\dfrac{\mathrm{d}^2 y}{\mathrm{d}x^2}+3x\dfrac{\mathrm{d}y}{\mathrm{d}x}+5y=0$通过变换\underline{\makebox[6em]{ }}可化为二阶常系数齐次线性微分方程.
		
		\vspace{1cm}
		\textbf{(以下拓展题目留给读者复习用)}
		\item 已知某曲线上任一点平分过该点的法线夹在两坐标轴之间的线段,则该曲线方程为\underline{\makebox[6em]{ }}.
		
		\item 如果$x(t)$是方程$x'=Ax$满足初始条件$x(t_0)=\eta$的解,那么$x(t)=$\underline{\makebox[6em]{ }}.
		
		\item 对于二阶线性齐次方程$x''+p(t)x'+q(t)x=0$(其中$p(t),q(t)$为连续函数),若有\underline{\makebox[6em]{ }}成立,则$x=t$是方程的解.
	\end{enumerate}
	
	\section{$\divideontimes$计算题(共24分,每题6分)}
	求下列微分方程的通解:
	\begin{enumerate}
		\item $(3x^2+6xy^2)\mathrm{d}x+(6x^2y+4y^3)\mathrm{d}y=0$
		\vspace{3cm}
		\item $x\dfrac{\mathrm{d}y}{\mathrm{d}x}=1+(\dfrac{\mathrm{d}y}{\mathrm{d}x})^2$
		\vspace{3cm}
		\item $x'+\sqrt{1-(x')^2}=0$
		\vspace{3cm}
		\item $x''-2x'+2x=te^t\cos t$
		\vspace{3cm}
		
		\textbf{(以下拓展题目留给读者复习用)}
		\item $x(\dfrac{\mathrm{d}y}{\mathrm{d}x})^2-2y\dfrac{\mathrm{d}y}{\mathrm{d}x}+4x=0$
		\vspace{3cm}
		\item $(xye^{\frac{x}{y}}+y^2)\mathrm{d}x-x^2e^{\frac{x}{y}}\mathrm{d}y=0$
		\vspace{3cm}
		\item $xx''+(x')^2=0$
		\vspace{3cm}
	\end{enumerate}

	\section{大题(第1-3题每题12分,第4题10分)}
	\begin{enumerate}
		\item (12分)$\divideontimes$(表述未必严谨)设$x_1(t),x_2(t)$是二阶齐次线性微分方程$x''+a_1(t)x'+a_2(t)x=0,t\in [0,1]$的任意两个解(其中$a_1(t),a_2(t)$在区间$[0,1]$上连续),由$x_1(t),x_2(t)$所构成的朗斯基行列式记为$W(t)$,试证:
		\begin{enumerate}
			\item[(1)] $W(t)$可以表示为\[W(t)=W(0)\cdot\exp\{-\int_{0}^{t} a_1(s)\mathrm{d}s\}.\]
			\item[(2)] 设$x_1(0)=x_2(0)=0$,求证$x_1(t),x_2(t)$在区间$[0,1]$上线性相关.
		\end{enumerate}
		\vspace{5cm}
		
		\item (12分)设$f(x)$为连续函数,且满足\[f(x)=xe^x-\int_{0}^{x} (x-t)f(t)\mathrm{d}x ,\]求$f(x)$.
		\vspace{5cm}
		
		\item (12分)设$(x,y)\in D:\left|x+1\right|\leqslant 1,\left|y\right|\leqslant 1$,求初值问题
		$\begin{cases}
			\dfrac{\mathrm{d}y}{\mathrm{d}x}=x+y^2\\
			y(0)=0
		\end{cases}$
		的解的存在区间,并求第三次近似解,给出在解的存在区间的误差估计.
		\vspace{5cm}
		
		\item (10分)$\divideontimes$考虑某种物质A经化学反应全部生成另一种物质B.设A的初始质量为$m_0$克,在$1$小时后生成B物质$g$克,试求:
		\begin{enumerate}
			\item[(1)] 经过3小时后,A物质起反应的量是多少?
			\item[(2)] 经过多少小时后,A物质中$75\%$的量已经起了反应?
		\end{enumerate}
		\vspace{5cm}
		
	\end{enumerate}
\end{document}
